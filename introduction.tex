\chapter{مقدمة}
الحمد لله الذي تتم بفضله الصالحات واصلي وأسلم على خير البشر نبينا محمد صلى الله عليه وسلم وعلى آله وصحبه الطيبين الطاهرين. اما بعد:
فلقد لاحظت خلال فترة تعلمي للغة الليتك افتقار المكتبة العربية الى كتاب مكتمل يشرح مبادي هذه اللغة بشكل منظم وبسيط. فعقدت العزم على تأليف هذه الكتاب وذلك لما رايته من الأهمية بمكان ان يتعلم القارئ العربي هذه اللغة والتي أصبحت اللغة البرمجية المحبوبة لدي الكثير من العلماء والباحثين والمهندسين. فمعظم جامعات العالم اليوم أصبحت تدرسها لطلابها فهي لغة  كتابة مشهورة , قوية ويمكن استخدامها في مجالات عدة. لذلك اردت ان يكون هذا الكتاب لبنة أولى للمساهمة في تعليم هذه اللغة. الذي امل ان يتم تدريسها في مدارسنا الحكومية في مراحل مبكرة كالمتوسطة والثانوية وذلك لان الأجيال الحالية على اطلاع واسع بالتقنية وخصوصا الكمبيوترية منها. فهي اصحبت قدرة على تعلمها والاستفادة منها دون مشقة. وبما ان هذا العمل بشري المصدر فانه لا يصل الى درجة الكمال لذلك ارجو ممن سنحت له الفرصة لقراءة هذا الكتاب ان يساهم في تحسين هذا الكتاب بارسال ملاحظاته الى ايميل المؤلف 
ahmad.alsaadi@kaust.edu.sa
 الذي يعدكم على انه سوف يعمل على اخذها في الاعتبار متى ما سنحت الفرصة لاصدار طبعة جديدة لهذا 
